\documentclass[a4paper]{article}
\usepackage{amsmath}
\usepackage{bm}
\usepackage{xcolor}
\usepackage{braket}
\begin{document}
\title{Chapter\;2}
\date{ }
\maketitle
\section{Occupation representaion}
To fully exploit statistic,we introduce occupation representation in Fock space whose vectors tells the numbers and types of particles in the system.We also introduce number operator $N(\bm{p})$ to count number of particles with momentum $\bm{p}$ in a occupation 'state'(this state is actually a vector in Fock space instead of Hilbert space).With this operator,we can avoid labelling particles.
\par eg. energy operator in discrete case is now $$\hat{H}=\underset{\bm{p}}{\sum}\omega_{\bm{p}}N(\bm{p})$$and momentum operator in discrete case is now $$\bm{P}=\underset{\bm{p}}{\sum}\bm{p}N(\bm{p})$$
\section{Formalism of Fock space}
In both discrete and continous case,we'll need certain creation and annihilation operators.We demand that
\subsection{Discrete}
\begin{align*}
	&[a_{\bm{p}},a_{\bm{p}'}]=0\\
	&[a^{\dagger}_{\bm{p}},a^{\dagger}_{\bm{p}'}]=0\\
	&[a_{\bm{p}},a^{\dagger}_{\bm{p}'}]=\delta_{\bm{p}\bm{p}'}\\
	&N(\bm{p})=a^{\dagger}_{\bm{p}}a_{\bm{p}}
\end{align*}
\subsection{Continous}
\begin{align*}
	&[a_{\bm{p}},a_{\bm{p}'}]=0\\
	&[a^{\dagger}_{\bm{p}},a^{\dagger}_{\bm{p}'}]=0\\
	&[a_{\bm{p}},a^{\dagger}_{\bm{p}'}]=\delta(\bm{p}-\bm{p}')\\
	&N(\bm{p})=a^{\dagger}_{\bm{p}}a_{\bm{p}}
\end{align*}
For both case,creation and annihilation obey certain rules under Poincare transformation:
for boost $\Lambda$,we have
\begin{align*}
	&U^{\dagger}(\Lambda)a^{\dagger}_{\bm{p}}U(\Lambda)=a^{\dagger}_{\Lambda^{-1}\bm{p}}\\
	&U^{\dagger}(\Lambda)a_{\bm{p}}U(\Lambda)=a_{\Lambda^{-1}\bm{p}}
\end{align*}
And for spacetime translation $T(x)=e^{iP\cdot x}$,we have:
\begin{align*}
	&e^{-iP\cdot x}a^{\dagger}_{\bm{p}}e^{iP\cdot x}=e^{-ip\cdot x}a^{\dagger}_{\bm{p}}\\
	&e^{-iP\cdot x}a_{\bm{p}}e^{iP\cdot x}=e^{ip\cdot x}a_{\bm{p}}
\end{align*}
\par All the sums in formular that show up in discrete case now must be replaced by integral:$$\underset{\bm{p}}{\sum}\rightarrow\int d^3\bm{p}$$
\par Vacuum state:the vaccum state contains no particle and it is unique that all the vacuums are one.Therefore,for any Lorentz transformation $\Lambda$ and sapcetime translation $T$,we must have $$\Lambda\ket{0}=\ket{0},T(a)\ket{0}=\ket{0}$$
Also,by convention,we set $\braket{0|0}=1$
\section{4-momentum creation and annihilation}
To find creation and annihilation that creates or annihilates a 4-momentum particle,we can derivate that:
\begin{align*}
	&\alpha^{\dagger}(p)=(2\pi)^{\frac{3}{2}}\sqrt{2\omega_{\bm{p}}}a^{\dagger}_{\bm{p}}\\
	&\alpha(p)=(2\pi)^{\frac{3}{2}}\sqrt{2\omega_{\bm{p}}}a_{\bm{p}}
\end{align*}
\section{Expanding an arbitrary state}
say $\Psi$ is a state in Fock space,then we can expand it in series:$$\Psi=\psi_0\ket{0}+\int d^3\bm{p}\,\psi(\bm{p})\ket{\bm{p}}+\frac{1}{2!}\int d^3\bm{p}_1 d^3\bm{p}_2\,\psi(\bm{p}_1,\bm{p}_2)\ket{\bm{p}_1,\bm{p}_2}+\cdots
$$where the coefficients are designed to kill permutation symmetry.
\par Of course and the completeness relationship$$1=\sum_{N=0}^{\infty}\frac{1}{N!}\int\frac{d^3\bm{p}_1}{(2\pi)^32E_{\bm{p}_1}}\cdots\frac{d^3\bm{p}_N}{(2\pi)^32E_{\bm{p}_N}}\ket{p_1,p_2\cdots p_N}\bra{p_1,p_2\cdots p_N}$$
\end{document}