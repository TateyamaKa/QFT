\documentclass[a4paper]{article}
\usepackage{amsmath}
\usepackage{bm}
\usepackage{xcolor}
\usepackage{braket}
\begin{document}
	\title{Chapter\;1}
	\date{ }
	\maketitle
	\section{Minkowski space measure}
	$\int d^4p$ is Lorentz invariant and we consider a relavitistic particle that satisfies $p^\mu p_\mu = p^2 = m^2$,where $m$ is the mass of the particle.We now consider the measure along the hypersurface where the particle lives on and $p^0>0$.The measure becomes
	\begin{align*}
		\int d^4p\,\delta(p^2-m^2)\theta(p^0)&=\int d^4p\,\theta(p^0)\delta((p^0)^2-\omega_{\bm{p}}^2)\\
		&=\int d^4p\,\theta(p^0)(\frac{\delta(p^0-\omega_{\bm{p}})}{2\omega_{\bm{p}}}+\frac{\delta(p^0+\omega_{\bm{p}})}{2\omega_{\bm{p}}})\\
		&=\int d^3\bm{p}\times\frac{1}{2\omega_{\bm{p}}}
	\end{align*}
	Thus,for a relavitistic particle,a proper Lorentz invariant measure is$$\textcolor{green}{\int\frac{d^3\bm{p}}{2\omega_{\bm{p}}}\, , \,\omega_{\bm{p}}=\bm{p}^2+m^2}$$
	\section{Normalized 4-momentum state}
	We need to construct a 4-momentum state $\ket{p}$ such that$$\frac{1}{(2\pi)^3}\int\frac{d^3\bm{p}}{2\omega_{\bm{p}}}\ket{p}\bra{p}=\bm{1}$$A resonable choice will be $$\textcolor{green}{\ket{p}=(2\pi)^{\frac{3}{2}}\sqrt{2\omega_{\bm{p}}}\ket{\bm{p}}}$$
	check:$$\frac{1}{(2\pi)^3}\int\frac{d^3\bm{p}}{2\omega_{\bm{p}}}\ket{p}\bra{p}=\int d^3\bm{p}\,\ket{\bm{p}}\bra{\bm{p}}=\bm{1}$$
	\section{Violation of causality}
	Let a particle initially centered at x=0:$$\braket{\bm{x}|\psi}=\delta(\bm{x})$$Involve it by Schrodinger equation,the probability finding a particle sitting at $\bm{x}$ and time $t$ will be determined by
	\begin{align*}
		\bra{\bm{x}}exp(-i\hat{H}t)\ket{\psi}&=\int d^3\bm{p}\,\bra{\bm{x}}e^{-i\hat{H}t}\ket{\bm{p}}\braket{\bm{p}|\psi}\\
		&=\int d^3\bm{p}\,\bra{\bm{x}}e^{-i\hat{H}t}\ket{\bm{p}}\int d\bm{x}'\braket{\bm{p}|\bm{x}'}\braket{\bm{x}'|\psi}
	\end{align*}
	But we've already know
	\begin{align*}
		\braket{\bm{x}|\bm{p}}&=(2\pi)^{-\frac{3}{2}}e^{i\bm{p}\cdot\bm{x}}\\
		e^{-i\hat{H}t}\ket{\bm{p}}&=e^{-\omega_{\bm{p}}t}\ket{\bm{p}}\\
		\braket{\bm{x}'|\psi}&=\delta(\bm{x}')
	\end{align*}
	With these,we continue the calculation:
	\begin{align*}
		&=\int d^3\bm{p}e^{-i\omega_{\bm{p}}t}(2\pi)^{-\frac{3}{2}}e^{i\bm{p}\cdot\bm{x}}\int d\bm{x}'(2\pi)^{-\frac{3}{2}}\delta(\bm{x}')\\
		&=\int \frac{d^2\bm{p}}{(2\pi)^3}e^{-i\omega_{\bm{p}}t+i\bm{p}\cdot\bm{x}}\\
		&=-\frac{i}{(2\pi)^2r}\int_{-\infty}^{+\infty}dp\,pe^{ipr-i\omega_{\bm{p}}t}
	\end{align*}
	where $r=|\bm{x}|$.The above expression does not vanish outside the lightcone(i.e. $r>ct$),which contradicts relativity.
	\section{A demand for multiparticle theory}
	Consider localizing a single particle.By uncertainty principle:$$\Delta p\cdot\Delta x\sim\hbar$$as $\Delta x$ gets closer to 0,$\Delta p$ increases dramatically and thus energy.The energy will be so huge that it can make pair productions,and no more single particle.
\end{document}